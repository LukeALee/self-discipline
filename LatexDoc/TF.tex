% 导言区
\documentclass[UTF8]{article}%book ,report ,letter
\usepackage{diagbox}
\usepackage{ctexcap}%采用中文标题宏包(标题是中文的)
\usepackage{graphicx}%图片包
\usepackage{color}%彩色文本
\usepackage{amsmath}
\usepackage{caption2}
\usepackage{enumerate}
\usepackage{hyperref} %超链接包
\usepackage{amsmath}
\usepackage{hyperref} %超链接包
%覆盖超链接红框
\hypersetup{
	colorlinks=true,
	linkcolor=black
}
\usepackage{minted}
\usepackage{listings}
\usepackage{xcolor}
\definecolor{mygreen}{rgb}{0,0.6,0}  
\definecolor{mygray}{rgb}{0.5,0.5,0.5}  
\definecolor{mymauve}{rgb}{0.58,0,0.82}  
\lstset{ %  
	numbers=left,
	backgroundcolor=\color{white},   % choose the background color; you must add \usepackage{color} or \usepackage{xcolor}  
	basicstyle=\footnotesize,        % the size of the fonts that are used for the code  
	breakatwhitespace=false,         % sets if automatic breaks should only happen at whitespace  
	breaklines=true,                 % sets automatic line breaking  
	captionpos=bl,                    % sets the caption-position to bottom  
	commentstyle=\color{mygreen},    % comment style  
	deletekeywords={...},            % if you want to delete keywords from the given language  
	escapeinside={\%*}{*)},          % if you want to add LaTeX within your code  
	extendedchars=true,              % lets you use non-ASCII characters; for 8-bits encodings only, does not work with UTF-8  
	frame=single,                    % adds a frame around the code  
	keepspaces=true,                 % keeps spaces in text, useful for keeping indentation of code (possibly needs columns=flexible)  
	keywordstyle=\color{blue},       % keyword style  
	%language=Python,                 % the language of the code  
	morekeywords={*,in},            % if you want to add more keywords to the set  
	numbers=left,                    % where to put the line-numbers; possible values are (none, left, right)  
	numbersep=5pt,                   % how far the line-numbers are from the code  
	numberstyle=\tiny\color{mygray}, % the style that is used for the line-numbers  
	rulecolor=\color{black},         % if not set, the frame-color may be changed on line-breaks within not-black text (e.g. comments (green here))  
	showspaces=false,                % show spaces everywhere adding particular underscores; it overrides 'showstringspaces'  
	showstringspaces=false,          % underline spaces within strings only  
	showtabs=false,                  % show tabs within strings adding particular underscores  
	stepnumber=1,                    % the step between two line-numbers. If it's 1, each line will be numbered  
	stringstyle=\color{orange},     % string literal style  
	tabsize=2,                       % sets default tabsize to 2 spaces  
	%title=myPython.py                   % show the filename of files included with \lstinputlisting; also try caption instead of title  
}

\title{\heiti Tensorflow}
\author{\kaishu LukeAlanLee}
\date{\today}

\begin{document}
	\tableofcontents
	\maketitle
	
	\section{简介}
		\subsection{推荐配置}
			\begin{itemize}
				\item windows: 10-1809
				\item python:3.7.0
				\item cuda:10.0
				\item cudnn:7.4.2
				\item tensorflow-gpu:1.31.1
			\end{itemize}
		\subsection{计算图}
			\begin{itemize}
				\item 定义常量:tf.constant()
				\item 定义会话:sess=tf.Session(), 关闭会话:sess.close(),可以使用with 管理会话(with tf.Session() as sess)
				\item 运行会话:sess.run()
				\item 默认会话:tf.InteractiveSession(),此时可以使用eavl()直接运行张量而不用显示调用会话
				\item 指定运行设备:tf.device(),CPU/GPU
				\item 张量类型:
				\begin{description}
					\item 常量:tf.constant()
					\item 变量:tf.Variable()
					\item 占位符:tf.placeholder(),通常和feed\_dict一起使用来输入数据
				\end{description}
				\begin{description}
					\item[举例:] tf.zeros(),tf.zeros\_like(),tf.ones(),tf.ones\_like(),tf.eye(n)
					\item[等差数列:] tf.linespace(start,stop,nums),tf.range(start,limit,delta)
					\item[正态分布:] tf.random\_normal([rows,cols],mean,stddev,seed)
					\item[截尾正态分布:] tf.truncated\_normal([rows,cols],mean,stddev,seed)
					\item[伽马分布:] tf.random\_uniform([rows,cols],maxval,seed)
					\item[裁剪张量:] tf.random\_crop(t\_radom,[rows,cols],seed)
					\item[随机重排:] tf.random\_shuffle(t\_radom)沿t\_radom第一维随机排列张量
					\item[设置随机种子:] tf\_set\_tandom\_seed()相同种子保证多次运行获得相同随机数(种子只能设为整数值)
					\item[变量别名:] tf.Variable(tf.zeros[100],name='biases')
					\item[初始化变量的方法:] 使用常量、已定义的变量都可以初始化新的变量;必须初始化所有变量 tf.global\_variables\_initializer()
					\item[数据类型转换:] tf.cast(var,dtype)
					\item[使用Saver类来保存变量:]  saver=tf.Saver()
					\item[定义placeholder并使用feed\_dict输入:] 举例如下:
				\end{description}
				\item \lstinputlisting[language=Python, title=tfTest.py]{codes/tfTest.py}  	
			\end{itemize}
			\center
			%数据类型说明
			\begin{tabular}[c]{|l|l|}
				\hline
				\multicolumn{2}{|c|}{tensorFlow数据类型}\\
				\hline
				数据类型 & tensorFlow类型 \\
				\hline
				DT\_FLOAT & tf.float32  \\
				\hline
				DT\_DOUBLE & tf.float64 \\
				\hline
				DT\_int8 & tf.int8  \\
				\hline
				DT\_UINT & tf.uint8  \\
				\hline
				DT\_STRING & tf.string  \\
				\hline
				DT\_BOOL & tf.bool  \\
				\hline
				DT\_COMPLEX64 & tf.complex64  \\
				\hline
				DT\_QINT32 & tf.qint32  \\
				\hline
			\end{tabular}
			%运算类型
			\begin{itemize}
				\item 运算
				\begin{description}
					\item[加法运算:]tf.add()
					\item[矩阵乘法:] tf.mutmul()
					\item[按元素相除] *
					\item[矩阵乘标量] tf.scalar\_mul(2,A)
					\item[按元素相除] tf.div()
					\item[按元素取余] tf.mod()
					\item[整数张量除法] tf.truediv(a,b),先将整数转为浮点类,然后再执行按位除
				\end{description}
			\item 数据流图-TensorBoard
				\begin{description}
					\item[捕获时间序列变化:] tf.summary.scalar()
					\item[捕获输出分布:]  tf.summary.histogram()
					\item[捕获所有摘要:] tf.merge\_all\_summaries()
					\item[保存摘要:] tf.summary.Filewriter('logdir',sess.graph)
					\item[tensorBoard:] tensorboard --logdir='logdir'
				\end{description}
			\end{itemize}
		\subsection{从文件读取数据}
			\flushleft
			\begin{enumerate}
				\item 创建文件名列表:使用字符串张量保存或使用files=tf.train.match\_filenams\_once('filename')
				\item 创建文件队列:之后使用tf.train.string\_input\_producer(files)创建文件队列
				\item 创建由换行符分隔的文件行的读取器:tf.TextLineReader()
				\item 从tensor列表中取出tensor放入文件名队列:input\_queue=tf.train.slice\_input\_producer()
				\item 从文件名队列中提取tensor放入文件队列:image\_batch, label\_batch = tf.train.batch(input\_queue, batch\_size=batch\_size, num\_threads=2, capacity=64)
				\item 线程协调器: coord=tf.train.Coordinator() ,用来管理之后在Session中启动的所有线程;
				\item 线程协调器方法:
				\begin{itemize}
					\item coord.should\_stop()来查询是否应该终止所有线程,当文件队列(queue)读取完毕,抛出一个OutofRangeError的异常,此时应该结束左右线程
					\item 使用coord.request\_stop()来发出终止所有线程的命令
					\item 使用coord.join(threads)把线程加入主线程,等待threads结束\textbf{}
				\end{itemize}
				\item 启动线程:tf.train.start\_queue\_runners(coord=coord),由多个或单个线程(使用coord作为线程管理器),按照设定规则,把文件读入Filename Queue中.
				\item Example
			\end{enumerate}

	
	\section{回归}
		\subsection{损失函数}
			\begin{enumerate}
				\item 平方损失:  tf.square()
				\item 滑动平均:   tf.reduce(tf.square())
				\item 交叉熵:  
				\begin{minted}
				[mathescape,
				linenos,
				numbersep=5pt,
				gobble=2,
				frame=lines,
				framesep=2mm
				]{Python}
				entropy=tf.nn.softmax_cross_entorypy_with_logits()
				loss=tf.reduce_mean(entropy)
				\end{minted}
		
				
			\end{enumerate}
		\subsection{正则化}
			\begin{enumerate}
				\item L1   tf.reduce\_sum()
				\item L2  tf.nn.l2\_loss()
			\end{enumerate}
		\subsection{优化器}
			\begin{enumerate}
				\item \color{red}GD:  \color{black} tf.train.GradientDescentOptimizer(learning\_rate)
				\item \color{red}momenttum: \color{black}  tf.train.MomentumOptimizer()
				\item \color{red}Adadleta:  \color{black} tf.train.AdadletaOptimizer()
				\item \color{red}RMSprop: \color{black}  tf.train.RMSpropOptimizer()
				\item \color{red}Adagrad: \color{black}  tf.train.AdagradOptimizer()
				\item \color{red}AdagradDA: \color{black}  tf.train.AdagradDAOptimizer()
				\item \color{red}Ftrl:  \color{black} tf.train.FtrlOptimizer()
				\item \color{red}ProximalGD: \color{black}  tf.train.ProximalGradientDescentOptimizer()
				\item \color{red}ProximalAdagrad: \color{black}  tf.train.ProximalAdagradOptimizer()
				\item 设置指数衰减学习率  tf.train.ecponential\_decay()
				\lstinputlisting[language=Python, title=exponentialDecayExample.py]{codes/exponentialDecayExample.py}  	
			\end{enumerate}
	
	\section{神经网络}
	\subsection{激活函数}
	\subsection{构建网络}
		\begin{enumerate}
		\item threshold:
		\begin{minted}[mathescape,
		linenos,
		numbersep=5pt,
		gobble=2,
		frame=lines,
		framesep=2mm]{Python}
		def threshold (x):
			cond = tf.less(x, tf.zeros(tf.shape(x), dtype = x.dtype))
			out = tf.where(cond, tf.zeros(tf.shape(x)), tf.ones(tf.shape(x)))
			return out		
		\end{minted}
		\item Sigmoid:
		\begin{minted}[mathescape,
		linenos,
		numbersep=5pt,
		gobble=2,
		frame=lines,
		framesep=2mm]{Python}
		
		out = tf.sigmoid(h)	
		\end{minted}
	
		\item tanh:
		\begin{minted}[mathescape,
		linenos,
		numbersep=5pt,
		gobble=2,
		frame=lines,
		framesep=2mm]{Python}
		
		out = tf.tanh(h)	
		\end{minted}

		\item ReLu:
		\begin{minted}[mathescape,
		linenos,
		numbersep=5pt,
		gobble=2,
		frame=lines,
		framesep=2mm]{Python}
		
		out = tf.nn.relu(h)	
		\end{minted}
		
		\item Softmax:
		\begin{minted}[mathescape,
		linenos,
		numbersep=5pt,
		gobble=2,
		frame=lines,
		framesep=2mm]{Python}
		
		#$y_i=\frac{\exp(x_i)}{\sum_j\exp(x_j)}$
		out = tf.nn.softmax(h)	
		\end{minted}
		\end{enumerate}
	\subsection{卷积神经网络}
	\begin{enumerate}
		\item 添加卷积层:
		\begin{minted}[mathescape,
		linenos,
		numbersep=5pt,
		gobble=2,
		frame=lines,
		framesep=2mm]{Python}
		
		tf.nn.conv2d(input ,filter,strides,padding,
			use_cudnn_on_GPU=None,data_format=None,name=None)
		\end{minted}
		\item 池化层
		\begin{itemize}
			\item 最大池化
		\begin{minted}[mathescape,
		linenos,
		numbersep=5pt,
		gobble=2,
		frame=lines,
		framesep=2mm]{Python}
		
		tf.nn.max_pool(value,kszie,strides,padding,data_format='NHWC',name=None)
				\end{minted}
		\end{itemize}
		\begin{minted}[mathescape,
		linenos,
		numbersep=5pt,
		gobble=2,
		frame=lines,
		framesep=2mm]{Python}
		
		out = tf.nn.relu(h)	
		\end{minted}
	\end{enumerate}
	\subsection{图片相关}
		\begin{enumerate}
			\item tf.gfile.FastGFile(path,decodestyle) 
			\begin{description}
				\item[函数功能:] 实现对图片的读取。 
				\item[函数参数:] (1)path:图片所在路径 (2)decodestyle:图片的解码方式。(‘r’:UTF-8编码; ‘rb’:非UTF-8编码)
			\end{description}
			
			
		\end{enumerate}
\end{document}