% 导言区
\documentclass[UTF8]{article}%book ,report ,letter
\usepackage{geometry}%页边距设置
\geometry{a4paper,scale=0.9}
\usepackage{ctex}%引入中文包,使得中文可以正常显示
\usepackage{diagbox}
\usepackage{algorithm2e,setspace}


\title{\heiti python}
\author{\kaishu 李晨辉}
\date{\today}


% 正文区(文稿区)
\begin{document}
	\maketitle %让头部内容在正文区显示
	%\begin{verbatim}
	%之间的文本将直接打印
	%\end{verbatim} 
	\flushleft
	\begin{enumerate}
		\item 生成器与迭代器的区别
			\begin{description}
				\item [a]. 有iter()和next()魔法方法的对象,都是迭代器(可以为你的类添加迭代器行为);
				\item[b.] 生成器是一个用于创建迭代器的工具,它们的写法类似标准的函数,但当它们要返回数据时会使用yield语句。每次对生成器调用next()时,它会从上次离开位置恢复执行
				\item[c.] 用生成器来完成的操作同样可以用基于类的迭代器来完成,但生成器的写法更为紧凑,因为它会自动创建 iter() 和 next() 方法。
				\item[d.] 局部变量和执行状态会在每次调用之间自动保存, 当生成器终结时,它们还会自动引发 StopIteration。
			\end{description}
		\item Python中,函数名为什么可以当作参数用?
		\begin{itemize}
			\item python中函数是第一等对象,第一等对象的一般特征:
				\begin{description}
					\item[a.] 运行时(runtime)创建
					\item[b.] 将变量或者元素赋值在一个数据结构当中
					\item[c.] 可以作为一个参数传递给一个函数
					\item[d.] 可以作为函数的结果返回
				\end{description}
			\item python 中一切皆对象,int是对象,函数是对象,class 也是一种对象,跟其它对象一样是最终继承自PyObject,函数可以像任何对象那样进行赋值、传递、名字重绑定、赋值、装进容器、垃圾回收
			
		\end{itemize}
	\end{enumerate}
	\center
	\begin{tabular}[c]{|c|c|c|}
		\hline
		\multicolumn{3}{|c|}{如何追女孩的建议}\\
		\hline 
		\diagbox{财气}{建议}{颜值} & 帅& 挫\\
		\hline
		有钱花 & 没啥好说的 & 没啥好说的 \\
		\hline
		叮当响 & 没啥好说的 & 没啥好说的  \\
		\hline
	\end{tabular}
	\begin{table}[hb] 
		\caption[短标题]{装模做样的表格}
		\begin{center}
			\begin{tabular}{c}
				\hline
				这是一个表格\\
				\hline
			\end{tabular}
		\end{center}	
	\end{table}
			%伪代码
	\begin{algorithm}
		\setstretch{1.35}
		\SetAlgoLined
		\KwData{this text}
		\KwResult{how to write algorithm with \LaTeX2e }
		initialization\;
		\For{not at end of this document}{
			read current\;
			\eIf{understand}{
				go to next section\;
				current section becomes this one\;
			}{
				go back to the beginning of current section\;
			}
		}
		\caption{How to write algorithms}
	\end{algorithm}

	你好,\LaTeX
\end{document}
